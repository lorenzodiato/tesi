\chapter{Conclusioni}
\label{chap:conclusioni}

Nella prima versione di OVL Dashboard io ed il mio collega siamo riusciti ad ottenere un versione completa e funzionante dal punto di vista della gestione delle connessioni e della comunicazione real-time sulla piattaforma.
Innanzitutto, si è deciso di adottare React come libreria principale per la realizzazione dell'interfaccia grafica. Questa soluzione ha permesso di suddividere l’applicazione in componenti riutilizzabili ed estendibili, migliorando l’efficienza dell'attività di sviluppo e consentendo una facile manutenzione del codice. Infine, per la gestione della logica applicativa, l’utilizzo di Redux ha consentito di gestire in modo facile la gestione dello stato dell'applicazione, contribuendo a rendere l’applicazione più modulare e a prevenire l’introduzione di alcuni tipi di bug e di codice ridondante. È stato fondamentale comprendere in modo completo tutti i punti di forza di React al fine di sfruttarlo a pieno e rendere la web application intuitiva e performante. 

\section{Sviluppi futuri}
\label{sec:sviluppi futuri}
Gli sviluppi futuri della piattaforma sono molteplici. Dal punto di vista delle funzionalità si punta a trasformare la dashboard in modo da renderla un elemento fondamentale nelle attività mediche quotidiane. In particolare occorre implementare la comunicazione real-time, non solamente come chiamato vocale o testuale, ma anche come streaming video. Questo permetterebbe un'eventuale consulenza tra medici con l'ausilio aggiuntivo del video in tempo reale per diagnosi di soggetti sotto osservazione. Ampliando ancora di più la visione di utilizzo di questo strumento è possibile equipaggiare i mezzi di soccorso medico con dispositivi dotati di connettività mobile per collegarsi alla dashboard ed ottenere un'opinione di uno specialista su un particolare caso clinico.
Inoltre, verrà aggiunta la possibilità da parte degli utenti amministratori di creare e gestire istanze di macchine virtuali grazie alle SDK del servizio EC2 offerte da AWS.
Dal punto di vista della sicurezza e del monitoraggio delle attività svolte all'interno della dashboard, viene utilizzata una blockchain privata basata su una rete Ethereum\footnote{Ethereum: \url{https://www.ethereum.org/}}. All'interno della blockchain, attualmente già implementata, verranno scritte e mantenute nel tempo tutte le operazioni eseguite dagli utenti all'interno di OVL Dashboard, garantendo la sicurezza delle informazioni. L'adozione di questa  tecnologia permette inoltre di avere uno storico, presentato sotto forma di log all'interno della piattaforma visibile solamente dall'account dell'azienda.
