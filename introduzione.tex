\chapter{introduzione}
\label{chap:introduzione}

\section{Ambito del progetto}
\label{sec:ambito del progetto}

Il presente lavoro ha come oggetto lo studio dello sviluppo e della progettazione della piattaforma OVL Dashboard, un portale che permette la gestione e il monitoraggio di una rete virtuale di dispositivi situati in diverse strutture sanitarie su tutto il territorio italiano. Questo progetto è stato realizzato all’interno dell’azienda Themis, durante il mio stage curricolare, ed ha richiesto circa tre mesi per la sua realizzazione.
Il sito è accessibile da chiunque possegga un account registrato alla piattaforma, la cui creazione avviene tramite invito. Un utente registrato può essere in grado di accedere alle macchine collegate alla rete OVL, modificarne le proprietà o interagire con altri utenti in base ai permessi da esso posseduti.
Ho scelto dall’elenco delle proposte di stage disponibili proprio questa in quanto sono sempre interessato alle tecnologie che permettono la migrazione in cloud di servizi di uso quotidiano. Inoltre, questo stage mi offriva la possibilità di applicare concretamente le mie competenze di full-stack developer e svilupparle in un contesto lavorativo.
Per la realizzazione di tale progetto ho utilizzato diverse tecnologie tra le quali JavaScript in particolare la libreria ReactJS, NodeJS, MySQL, MongoDB, Git.

\section{Descrizione dell'azienda}
\label{sec:descrizione dell'azienda}

La società Themis s.r.l. nasce nell’incubatore del Politecnico di Torino nell’anno 2004, come evoluzione da realtà professionali e da altre società che hanno come filo conduttore l’attività d’ingegneria del software, nata come UNO s.r.l. nel 1981. Dalla sua nascita, Themis interviene nell’organizzazione e management di progetti per l’automazione di sistemi complessi su specifiche che non hanno riscontro in soluzioni pronte nel settore I.T. e nella  raccolta, distribuzione e gestione di dati e condivisione delle risorse, integrando i propri prodotti con le tecnologie più innovative che utilizzano il cloud come naturale centro nodale delle proprie piattaforme.